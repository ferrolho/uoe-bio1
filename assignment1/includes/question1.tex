\section{Question}

\subsection{What is the name of the disease you have selected?}

I have selected \textit{Huntington Disease}, also known as \textit{Huntington Chorea}.

\textit{HD} is a genetic brain disorder which causes jerky movements, emotional problems, and loss of cognition.

% - - - - - - - - - - - - - - - - - - - - - - - - - - -

\subsection{Explain why it is thought there is a genetic basis for this disease.}

A DNA segment known as a \href{https://ghr.nlm.nih.gov/art/large/repeatexpansion.jpeg}{CAG trinucleotide repeat} has been consistently detected in people who have been subjected to \textit{Molecular Genetic Testing} and suffer from \textit{HD}.

It is believed that an increase of the \textit{CAG segment} length causes \textit{huntingtin} proteins to be longer. Furthermore, these abnormal proteins get split into toxic segments which accumulate in neurons. This compromises neurons' normal behaviour, and might lead to their death.

The manifestation of these events damage areas of the brain, thus originating the symptoms of a person with \textit{HD}.

% - - - - - - - - - - - - - - - - - - - - - - - - - - -

\subsection{What is the human name for the gene that is thought to be involved?}

The official symbol is \textit{HTT}. It's official full name is \textit{huntingtin}. Furthermore, it's gene ID in the \href{http://www.ncbi.nlm.nih.gov/gene/}{NCBI Gene Database } is \textit{3064}.

% - - - - - - - - - - - - - - - - - - - - - - - - - - -

\subsection{Is this gene known by any other names? Whether yes or no, explain how you investigated this.}

As one can see in the \href{https://www.ncbi.nlm.nih.gov/gene/3064}{NCBI \textit{huntingtin} page}, the gene is also known as \textit{HD}, and \textit{IT15}.

% - - - - - - - - - - - - - - - - - - - - - - - - - - -

\subsection{Is this gene present in a model organism such as the mouse or the fruit fly?}

Yes, it is. The \textit{NCBI} database contains \href{https://www.ncbi.nlm.nih.gov/gene/?Term=ortholog_gene_3064[group]}{orthologs of numerous species}, including \href{https://www.ncbi.nlm.nih.gov/gene/15194}{Mus musculus} and \href{https://www.ncbi.nlm.nih.gov/gene/43392}{Drosophila melanogaster}.

\newpage
