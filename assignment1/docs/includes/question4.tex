\section{Question}

\large{\textbf{Now look at the following database:}}

\large{\textbf{\href{http://www.kazusa.or.jp/codon/}{Kazusa - Codon Usage Database}}}

\large{\textbf{The \textit{Codon Usage Database} lists the frequency which each codon is used in a species (different species prefer different codons). Sequences which have too many rarer codons result in slowing down transcription and inhibition of protein expression - in extreme cases, rare codons are thought to introduce transcription errors when the rare tRNA is not available.}}

\large{\textbf{If you were to try and express your human cDNA sequence in yeast (\textit{Saccharomyces cerevisiae}), which codons in your sequence might cause problems for expression?}}

\textbf{Note there is no hard threshold, but generally codons with 1\% usage or less are considered rare.}

\bigskip

According to the \href{http://www.kazusa.or.jp/codon/cgi-bin/showcodon.cgi?species=4932}{\textit{Saccharomyces cerevisiae} DB page}, yeast have \textbf{6534504} codons. Below is a table with the frequency of each codon in the format \textit{[triplet] [frequency: \textbf{percentage}] ([number])}.

% - - - - - - - - - - - - - - - - - - - - - - - - - - -

\begin{center}
    \footnotesize
    \begin{tabu} to \textwidth{ | X[c] | X[c] | X[c] | X[c] | }
        \hline
        & & & \\
        UUU 2.61 (170666) & UCU 2.35 (153557) & UAU 1.88 (122728) & UGU 0.81  (52903) \\
        UUC 1.84 (120510) & UCC 1.42  (92923) & UAC 1.48  (96596) & UGC 0.48  (31095) \\
        UUA 2.62 (170884) & UCA 1.87 (122028) & UAA 0.11   (6913) & UGA 0.07   (4447) \\
        UUG 2.72 (177573) & UCG 0.86  (55951) & UAG 0.05   (3312) & UGG 1.04  (67789) \\
        & & & \\
        \hline
        & & & \\
        CUU 1.23  (80076) & CCU 1.35  (88263) & CAU 1.36  (89007) & CGU 0.64  (41791) \\
        CUC 0.54  (35545) & CCC 0.68  (44309) & CAC 0.78  (50785) & CGC 0.26  (16993) \\
        CUA 1.34  (87619) & CCA 1.83 (119641) & CAA 2.73 (178251) & CGA 0.30  (19562) \\
        CUG 1.05  (68494) & CCG 0.53  (34597) & CAG 1.21  (79121) & CGG 0.17  (11351) \\
        & & & \\
        \hline
        & & & \\
        AUU 3.01 (196893) & ACU 2.03 (132522) & AAU 3.57 (233124) & AGU 1.42  (92466) \\
        AUC 1.72 (112176) & ACC 1.27  (83207) & AAC 2.48 (162199) & AGC 0.98  (63726) \\
        AUA 1.78 (116254) & ACA 1.78 (116084) & AAA 4.19 (273618) & AGA 2.13 (139081) \\
        AUG 2.09 (136805) & ACG 0.80  (52045) & AAG 3.08 (201361) & AGG 0.92  (60289) \\
        & & & \\
        \hline
        & & & \\
        GUU 2.21 (144243) & GCU 2.12 (138358) & GAU 3.76 (245641) & GGU 2.39 (156109) \\
        GUC 1.18  (76947) & GCC 1.26  (82357) & GAC 2.02 (132048) & GGC 0.98  (63903) \\
        GUA 1.18  (76927) & GCA 1.62 (105910) & GAA 4.56 (297944) & GGA 1.09  (71216) \\
        GUG 1.08  (70337) & GCG 0.62  (40358) & GAG 1.92 (125717) & GGG 0.60  (39359) \\
        & & & \\
        \hline
    \end{tabu}
\end{center}

% - - - - - - - - - - - - - - - - - - - - - - - - - - -

By examining the table, one can see that codons with a frequency lower than 1\% are: \texttt{UGU}, \texttt{UGC}, \texttt{UAA}, \texttt{UGA}, \texttt{UCG}, \texttt{UAG}, \texttt{CGU}, \texttt{CUC}, \texttt{CCC}, \texttt{CAC}, \texttt{CGC}, \texttt{CGA}, \texttt{CCG}, \texttt{CGG}, \texttt{AGC}, \texttt{ACG}, \texttt{AGG}, \texttt{GGC}, \texttt{GCG}, \texttt{GGG}. Thus, these codons might introduce errors during transcription.

We can even see that \texttt{UAA} (0.11\%), \texttt{UAG} (0.05\%), and \texttt{UGA} (0.07\%) are very rare triplets (for yeast). All these three codons correspond to the \textbf{termination} codon - see \href{https://en.wikipedia.org/wiki/Genetic_code#RNA_codon_table}{RNA codon table}.

This is not a problem however, because despite the low percentage of these codons, there are thousands of each available, which by far cover the needs of the protein expression.

Another codon which might be considered \textit{rare} is \textbf{CGG}, but again that should not be a problem because the expression does not demand a number of that triplet greater than its frequency in yeast.

Having said that, suppose the frequency of \textbf{CGG} was much lower - so low that it could introduce transcription errors. One possible work around is to reverse engineer the \textit{mRNA}. \textbf{CGG} codes \textit{Arginine}, which is also coded by \textbf{CGU}, \textbf{CGC}, and \textbf{CGA} - which have a much higher frequency (check table above). One could replace \textbf{CGG} occurences in the \textit{mRNA} with one of those other three \textit{Arginine} encoding codons, and that should solve the transcription errors.

\newpage
